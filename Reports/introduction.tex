\begin{figure}[ht]
	\begin{center}
		\includegraphics*[trim=10mm 50mm 65mm 5mm,width=100mm]{tx_architecture}
	\end{center}
	\caption{System architecture.}
	\label{fig:sm}
\end{figure}
%\includegraphics[width=.8\textwidth,natwidth=297mm,natheight=210mm]{multi-bs-1.eps}

$\text{N}_\text{A}$ is the number of antennas in a single analog beam steering blocks.

Total number of transmit antennas is $\text{N}_\text{A}\times\text{N}_\text{RF}$

Number of baseband  chains at the transmitter is $\text{N}_\text{RF}$. This is also equal to the number of analog beam steering blocks.

Number of receive antennas is $\text{N}_\text{R}$

$\text{W}$ is the DFT matrix

$\text{E}_\text{F}$ is the analog precoder matrix in the frequency domain and $\text{E}$ is the analog precoder matrix in the time domain.

$\text{P}$ is the digital precoder matrix

$\text{H}$ is the received channel matrix of dimensions $\text{N}_\text{R} \times (\text{N}_\text{A}\times\text{N}_\text{RF})$

$\text{X}$ is the transmitted data.

$y$ is the received signal and $n$ is AWGN noise
\begin{equation}
y = \text{H}\text{E}_\text{F}\text{P}\text{X} + n. 
\label{eq:y}
\end{equation}

Matrix E can be represented as a product of two matrices, a block diagonal and a square matrix. The block diagonal one represents the configuration of the analog beam sterring blocks and the square matrix represents the cross connections of the hybrid beamforming system.
\begin{eqnarray}
\text{E}_\text{F} &=& \text{W} \times \text{E} \\
\text{E} &=& \text{E}_\text{BD} \times \text{E}_\text{S} \\
\text{E}_\text{BD} &=& \begin{bmatrix}
\begin{bmatrix} e^{j\phi_{1,1}}\\ e^{j\phi_{2,1}}\\ \vdots \\e^{j\phi_{\text{N}_\text{A},1}}\end{bmatrix} & 
\begin{bmatrix} 0\\ 0\\ \vdots \\0\end{bmatrix} & \cdots & \begin{bmatrix} 0\\ 0\\ \vdots \\0\end{bmatrix} \\
\begin{bmatrix} 0\\ 0\\ \vdots \\0\end{bmatrix} & \begin{bmatrix} e^{j\phi_{1,2}}\\ e^{j\phi_{2,2}}\\ \vdots \\e^{j\phi_{\text{N}_\text{A},2}}\end{bmatrix} & \cdots &
\begin{bmatrix} 0\\ 0\\ \vdots \\0\end{bmatrix} \\
 \vdots & \vdots & \ddots & \vdots  \\
 \begin{bmatrix} 0\\ 0\\ \vdots \\0\end{bmatrix} &  \begin{bmatrix} 0\\ 0\\ \vdots \\0\end{bmatrix} &  \cdots & 
 \begin{bmatrix} e^{j\phi_{1,\text{N}_\text{RF}}}\\ e^{j\phi_{2,\text{N}_\text{RF}}}\\ \vdots \\e^{j\phi_{\text{N}_\text{A},\text{N}_\text{RF}}}\end{bmatrix}
\end{bmatrix} \\
 \text{E}_\text{S} &=&
\begin{bmatrix}
\text{A}_{1,1}e^{j\theta_{1,1}} &  \cdots& \text{A}_{1,\text{N}_\text{RF}}e^{j\theta_{1,\text{N}_\text{RF}}} \\
\text{A}_{2,1}e^{j\theta_{2,1}} &  \cdots &\text{A}_{2,\text{N}_\text{RF}}e^{j\theta_{2,\text{N}_\text{RF}}} \\
&\ddots&\\
\text{A}_{\text{N}_\text{RF},\text{N}_\text{RF}}e^{j\theta_{\text{N}_\text{RF},\text{N}_\text{RF}}} &  \cdots &\text{A}_{\text{N}_\text{RF},\text{N}_\text{RF}}e^{j\theta_{\text{N}_\text{RF},\text{N}_\text{RF}}} 
\end{bmatrix}
\end{eqnarray}

$e^{j\phi_{m,n}}$ represents the phase shift provided by the $m_{th}$ phase shifter  in the $n_{th}$ analog beam steering block. $\text{A}_{p,n}$ and $\theta_{p,n}$ are respectively the gain and phase shift provided to the $p_{th}$ time domain data stream which is being transmitted by the $n_{th}$ analog beam steering block.

Things to find out
\begin{enumerate}
	\item For a given E matrix, a P matrix which maximizes the sum rate,
	\item Study the following topologies of $\text{E}_\text{S}$
	\begin{enumerate}
		\item A binary diagonal matrix,
		\item binary bidiagonal matrix,
		\item binary triangular matrix,
		\item diagonal matrix,
		\item bidiagonal matrix,
		\item triangular matrix,
		\item a fully connected matrix,
	\end{enumerate}
\end{enumerate}